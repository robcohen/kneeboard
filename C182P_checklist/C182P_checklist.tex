\documentclass{article}
\usepackage{geometry}
    \geometry{
        a5paper,
        portrait,
        % The Jeppesen plate appears to be closer to 0.25in. I think
        % that 0.5in is looking best for checklists. Compromising to
        % accomodate longer line lengths.
        margin=0.25in,
        rmargin=0.375in,
        % headsep is the separation between header and text. footskip is
        % the separation between baseline of last line of text and
        % baseline of footer. The default is a bit larger. Setting these
        % to the ~line height pleases me, aesthetically.
        headsep=\baselineskip,
        footskip=\baselineskip,
        includehead,
        includefoot
    }
% NASA says:
% > The horizontal spacing between characters should be 25% of the
% > overall size and not less than one stroke width.
%
% The microtype documentation says:
% > Letterspaced fonts for which settings don’t exist will be spaced out
% > by the default of 0.1 em [...]
% AND
% > The amount is specified in thousandths of 1 em [...]
%
% So, we're scaling the default spacing by 25%, and then converting to
% housandths of an em (0.1 * 1000 * .25).
\usepackage[letterspace=25]{microtype}
% Used for the checklist frames.
\usepackage[many]{tcolorbox}
% For the preflight checklist square.
\usepackage{amssymb}
% For finer control over multi-column layouts.
\usepackage{multicol}
% For finer control over headers and footers.
\usepackage{fancyhdr}
% Used for degree symbol.
\usepackage{gensymb}
% Used for PDF ToC links.
\usepackage{hyperref}
% For printing the creation date of the document.
\usepackage{datetime2}
% Used for drawing patterns (e.g., the striped emergency procedure background).
\usepackage{tikz}
\usetikzlibrary{patterns,patterns.meta}
% Used for performance charts.
\usepackage{booktabs}
% Used for performing math inline.
\usepackage{xfp}
% Used for formatting numbers.
\usepackage{siunitx}
\sisetup{
    math-rm=\symup,
    detect-all,
    group-minimum-digits=4,
    group-separator={,}
}
% Improves positioning of tables and figures.
\usepackage{float}

% This is a macro that formats the checklist items and adds a new line.
\def\checkitem#1#2{
    #1\dotfill#2

}

% Set the default font family to sans-serif.
\renewcommand{\familydefault}{\sfdefault}

% NASA says:
% > The vertical spacing between lines should not be smaller than 25-33%
% > of the overall size of the font.
\renewcommand{\baselinestretch}{1.25}

% Configure the header and footer.
\pagestyle{fancy}
\fancyhf{}
\fancyhead[L]{PA-28-181, Piper Archer II}
\fancyfoot[L]{v.\today}
\fancyfoot[R]{\thepage}

% We don't need numbered sections.
\setcounter{secnumdepth}{0}

\begin{document}

% Apply microtype tracking adjustments.
\lsstyle

% [...] you can say \raggedcolumns if you don’t want the bottom lines to
% be aligned. The default is \flushcolumns, so TEX will normally try to
% make both the top and bottom baselines of all columns align.
\raggedcolumns

% This is a macro to assist in the preflight checklist.
\def\todoitem#1{
    \item[$\square$] #1 \dotfill
}

\section{Preflight Checklist}

\subsection{Master Switch On}

\begin{itemize}
    \todoitem{Interior and exterior lights}
    \todoitem{Stall warning horn}
    \todoitem{Pitot heat}
\end{itemize}

\subsection{Walk Around}

\begin{itemize}
    \todoitem{Control surfaces and cables}
    \todoitem{Drain fuel tank sumps}
    \begin{itemize}
        \item[$\bullet$] Remove all water and sediment; verify proper fuel.
    \end{itemize}
    \todoitem{Propeller}
    \todoitem{Air inlets and alternator belt tension}
    \todoitem{Oil level (6-8 quarts)}
    \todoitem{No obvious oil or fuel leaks}
\end{itemize}

\subsection{Landing Gear}

\begin{itemize}
    \todoitem{Strut exposure ($\geq 4.5''$ for main, $\geq 3.25''$ for nose)}
    \todoitem{Visual inspection of tires}
    \todoitem{Visual inspection of brake blocks}
\end{itemize}

\subsection{Finishing Up}

\begin{itemize}
    \todoitem{Clean windshield}
    \todoitem{Prep cockpit}
    \todoitem{Engine times and Stratus}
\end{itemize}

% This is the box that surrounds the checklist items.
\definecolor{ceruleanblue}{rgb}{0.16, 0.32, 0.75}
\newtcolorbox{checklist}[1]{
    colback=white,
    colframe=ceruleanblue,
    fonttitle=\centering\bfseries,
    adjusted title={#1},
    sharpish corners,
    phantom=\phantomsection,
    add to list={toc}{subsection}
}
\definecolor{ao(english)}{rgb}{0.0, 0.5, 0.0}
\newtcolorbox{checklistg}[1]{
    colback=white,
    colframe=ao(english),
    fonttitle=\centering\bfseries,
    adjusted title={#1},
    sharpish corners,
    phantom=\phantomsection,
    add to list={toc}{subsection}
}

\phantomsection
\addcontentsline{toc}{section}{Normal Procedures}

\twocolumn

\begin{checklist}{Preflight}
    \begin{center}
        \emph{In-Cabin}
    \end{center}
    \checkitem{Documents}{ARROW}
    \checkitem{Tach/Hobbs}{Recorded}
    \checkitem{Control Lock}{Removed}
    \checkitem{Emergency Equipment}{Check}
    \checkitem{Magnetos}{Off}
    \checkitem{Alternate Static}{Closed}
    \checkitem{Circuit Breakers}{In}
    \checkitem{Electrical Equipment}{Off}
    \checkitem{Bat. Switch}{On}
    \checkitem{Fuel Quantity}{Set}
    \checkitem{Flaps}{Full}
    \checkitem{Avionics/Fan}{On,Fan,Off}
    \checkitem{Bat. Switch}{Off}
    \begin{center}
        \emph{Dirty Work}
    \end{center}
    \checkitem{Fuel Sump}{Both Clear}
    \checkitem{Dip Fuel}{Record}
    \checkitem{Engine Oil}{10-12 Qts}
    \checkitem{Engine Fuel Flush}{No Water}
    \checkitem{Tire Pres. Nose/Main}{49/42 PSI}
    \begin{center}
        \emph{Exterior Inspection}
    \end{center}
    \checkitem{Tiedowns/Chocks}{Out}
    \checkitem{Baggage Door}{Secured}
    \checkitem{Towbar}{Stowed}
\end{checklist}

\begin{checklist}{Before Start}
    \checkitem{Exterior Inspections}{Complete}
    \checkitem{Passenger Briefing}{Standard}
    \checkitem{Seats/Seat Belts}{Set, Secure}
    \checkitem{Parking Break}{Set}
    \checkitem{Circuit Breakers}{Check}
    \checkitem{Avionics}{Off}
    \checkitem{Fuel Selector}{Both}
    \checkitem{Cowl Flaps}{Open}
\end{checklist}


\begin{checklist}{Start}
    \checkitem{Carburetor Heat}{Cold}
    \checkitem{Throttle}{Open 1/2", Set}
    \checkitem{Propellor}{High RPM, Set}
    \checkitem{Mixture}{Rich, Set}
    \checkitem{Battery Master}{On}
    \checkitem{Beacon}{On}
    \checkitem{Ext. Lights}{On as Required}
    \checkitem{Prime}{As Required}
    \checkitem{Prop. Area}{Clear Prop}
    \checkitem{Ignition}{Start}
    \checkitem{Oil Pressure}{Green 30s/60s}
    \checkitem{Ammeter}{Check, On, Charge}
    \checkitem{Mixture}{Lean, Set}
    \checkitem{Avionics}{On, Set}
    \checkitem{Flaps}{Retract}
    \checkitem{Transponder}{ALT}
    \checkitem{Parking Break}{Off}
    \checkitem{Breaks}{Test}
\end{checklist}

\begin{checklist}{Ready to Taxi}
    \checkitem{Garmin Database}{Updated}
    \checkitem{ATIS}{Copied}
    \checkitem{Transponder}{Set}
    \checkitem{COM \& NAV}{Set}
    \checkitem{Initial Alt.}{Set}
    \checkitem{Initial Heading}{Set}
    \checkitem{Exterior Lights}{Set}
    \checkitem{Clearance}{Recieved}
\end{checklist}

% Mike Busch and John Deakin have made some interesting points about the
% engine run up process. We're running so rich, and at such low power on
% the ground, it's hard to identity any real problems with the ignition
% system outside of a dead spark plug, a dead mag, or severe spark plug
% fouling.
% TBD: Finish this.
% See: https://www.avweb.com/flight-safety/pelicans-perch-19putting-it-all-together/
\begin{checklist}{Engine Run-Up}
    \checkitem{Seats/Belts}{Secure}
    \checkitem{Cabin Doors}{Closed}
    \checkitem{Flight Controls}{Free \& Correct}
    \checkitem{Autopilot}{Check, Off}
    \checkitem{Flight Instruments\*\*}{Set}
    \checkitem{Fuel Quantity}{Check}
    \checkitem{Cowl Flaps}{Open}
    \checkitem{Fuel Selector}{Both}
    \tcblower
    \checkitem{Mixture}{Full Rich}
    % John Deakin says:
    % > The usual 1,700 RPM for running up most TCM engines (or 2,000
    % > RPM for most Lycomings) is NOT critical. I’ve seen pilots diddle
    % > and dawdle trying to get exactly 1,700 but all this does is heat
    % > the engine up for no good purpose. Plus or minus a couple
    % > hundred RPM won’t hurt a thing, so push it up to “about 1,700”
    % > or “about 2,000” and get on with it.
    % > [...]
    % > It is also becoming very clear that the mag check at low power
    % > (anything less than cruise power) is not very useful for
    % > catching problems; it’s nothing more than a quick check to catch
    % > major problems like severe plug fouling, a “hot mag,” or a dead
    % > plug, or cylinder. This was well-known in the big old radials,
    % > where mag checks are almost always performed at about 30″ MP,
    % > and up around 2,300 RPM (varies with model).
    \checkitem{Propellor}{High RPM}
    \checkitem{Throttle}{1700-2000 RPM}
    \checkitem{Oil Pressure/Temp}{Green}
    \checkitem{Cyl. Head Temp}{Green}
    \checkitem{Ammeter}{Check}
    \checkitem{Annunciators}{Check}
    \checkitem{Vacuum}{4.6-5.4 Hg.}
    \checkitem{Magnetos}{Check R \& L}
    \centering{
        (max drop 150; max $\Delta$ 50)
        \\
    }
    \checkitem{Propellor}{Cycle 3X}
    \checkitem{Carb Heat}{Hot}
    \checkitem{Throttle}{Idle}
    \checkitem{Throttle}{700 RPM}
    \checkitem{Carb Heat}{Cold}
    \checkitem{Mixture}{Lean for Taxi}
    \checkitem{Circuit Breakers}{In}
    \checkitem{Alternate Static}{Check}
\end{checklist}

\begin{checklist}{Before Takeoff}
    \checkitem{Doors \& Windows}{Secured}
    \checkitem{Carb. Heat}{Off}
    \checkitem{Flaps}{0-20\degree}
    \checkitem{Trim}{Set}
    \checkitem{Cowl Flaps}{Full Open}
    \checkitem{Lights}{As Req.}

    \begin{center}
        \emph{\hypertarget{departure-briefing}{Departure Briefing}}
    \end{center}

    % TODO: Include some prompt to think about any possibility of
    % tailwind, extreme crosswind, or extreme temperature inversion.
    \checkitem{\hyperlink{runway-length-table}{Takeoff Distance}}{Briefed}
    \checkitem{Terrain \& Obstacles}{Briefed}
    \checkitem{Takeoff Minimums}{Briefed}
    \checkitem{Departure Procedure}{Briefed}

    \begin{center}
        \emph{Abnormal Operations}
    \end{center}

    % TODO: What will we do in case of a fire? Who will fly in an
    % emergency?
    \checkitem{Rejected Takeoff}{Briefed}
    \checkitem{Engine Power Loss}{Briefed}
    \centering{
        (below \& above $\approx$ 600' AGL)
        \\
    }
\end{checklist}

\begin{checklist}{Takeoff}
    \checkitem{Confirm Runway}{\# Confirmed}
    \checkitem{Target Airspeed}{53-78 KIAS}
    \checkitem{Mixture}{Rich/Target EGT}
    \checkitem{Carb Heat}{Cold}
    \checkitem{Throttle}{Full}
    \checkitem{Rotate}{70 KIAS}
    \checkitem{Flaps}{Retract at 70 KIAS}

    \begin{tcolorbox}[boxsep=0mm,left=0mm,right=0mm,colframe=black,colback=black,sharpish corners] 
        \color{white}
        \centering {
            \textbf{IF I LOSE THE ENGINE,\\I WILL PUSH IMMEDIATELY!}
        }
    \end{tcolorbox}
\end{checklist}

\begin{checklist}{Enroute Climb}
    \checkitem{Target Airspeed}{87-96 KIAS}
    \checkitem{Power}{23"/2450 RPM}
    \checkitem{Prop}{As Req.}
    \checkitem{Mixture}{Rich}
    \checkitem{Cowl Flaps}{As Req.}
\end{checklist}

\begin{checklist}{Cruise}
    \checkitem{Target Airspeed}{87-96 KIAS}
    \checkitem{Power}{15"-23"/2200-2450 RPM}
    \checkitem{Prop}{As Req.}
    \checkitem{Mixture}{Leaned}
    \checkitem{Trims}{As Req.}
    \checkitem{Cowl Flaps}{As Req.}
\end{checklist}

\begin{checklist}{Descent}
    \checkitem{Fuel Selector}{Both}
    \checkitem{Cowl Flaps}{As Req.}
    \checkitem{Rudder Trim}{Reset}
    \checkitem{Mixture}{Rich}
    \checkitem{Carb Heat}{As Req.}
    \checkitem{Power}{As Req.}
    \checkitem{ATIS}{Copied} 
    \checkitem{Arrival \& Approach}{Briefed}
    \checkitem{Terrain \& Taxi}{Briefed}
    \checkitem{Specials}{Briefed}
\end{checklist}

\begin{checklistg}{Before Landing}
    \checkitem{Seat \& Belts}{Secure}
    \checkitem{Fuel Selector}{Both}
    \checkitem{Mixture}{Rich}
    \checkitem{Propellor}{High RPM}
    \checkitem{Cowl Flaps}{As Req.}
    \checkitem{Rudder Trim}{Neutralize}
    \checkitem{Ext. Lights}{As Req.}
    \checkitem{Pitot Heat}{As Req.}
\end{checklistg}

\begin{checklistg}{Normal Landing}
    \checkitem{Airspeed Flaps Up}{70-78 KIAS}
    \checkitem{Wing Flaps}{0 to 40\degree}
    \checkitem{Airspeed Flaps Down}{61-70 KIAS}
\end{checklistg}

\begin{checklistg}{After Landing}
    \checkitem{Flaps}{Full Retract}
    \checkitem{Cowl Flaps}{Open}
    \checkitem{Carb Heat}{Cold}
    \checkitem{Mixture}{Lean for Taxi}
    \checkitem{Lights}{As Required}
\end{checklistg}

\begin{checklistg}{Securing Aircraft}
    \checkitem{Hobbs \& Tach}{Record}
    \checkitem{Lights}{Off}
    \checkitem{Avionics}{Off}
    \checkitem{Throttle}{700 RPM}
    \checkitem{Mixture}{Idle Cutoff}
    \checkitem{Magnetos}{Off \& Pull Key}
    \checkitem{Master Switch}{Off}
    \checkitem{Position Plane}{Chocks}
    \checkitem{Cowl Flaps}{Closed}
    \checkitem{Parking Break}{Set}
\end{checklistg}

\begin{checklist}{V-Speeds}
    \checkitem{$V_{BG}$ flaps Up/Down}{70/65}
    \checkitem{$V_R$ (flaps 0\degree{}/25\degree{})}{60/50 KIAS}
    \checkitem{$V_{X}$ sea/10K}{59/63 KIAS}
    \checkitem{$V_{Y}$ sea/10K}{80/63 KIAS}
    \checkitem{$V_{A}$}{89-110 KIAS}
    \checkitem{$V_{S_{0}}$/$V_{S_{1}}$}{48/53 KIAS}
\end{checklist}

\onecolumn

\newtcolorbox{checklist_emerg}[1]{
    enhanced,
    title style={
        pattern={Lines[angle=60,distance=32pt,line width=16pt]},
        pattern color=black!75,
    },
    colback=white,
    colframe=black,
    fonttitle=\centering\bfseries,
    adjusted title={#1},
    sharpish corners,
    phantom=\phantomsection,
    add to list={toc}{subsection}
}

\phantomsection
\addcontentsline{toc}{section}{Emergency Procedures}

\twocolumn

\begin{checklist_emerg}{Electrical Fire\\(Smoke in Cabin)}
    \checkitem{Master switch}{off}
    \checkitem{Avionics master}{off}
    \checkitem{Electrical switches}{off}
    \begin{center}
        \textbf{If no smoke:}
    \end{center}
    \checkitem{Circuit breakers}{note tripped}
    \checkitem{Circuit breakers}{off}
    \checkitem{Master switch}{on}
    \begin{center}
        \textbf{If no smoke:}
    \end{center}
    \checkitem{Avionics master}{on}
\end{checklist_emerg}

\begin{checklist_emerg}{Alternator Failure}
    \checkitem{Verify failure}{}
    \checkitem{Reduce electrical load as much as possible}{}
    \checkitem{Alt circuit breakers}{check}
    \checkitem{Alt switch}{off, wait, then on}
    \begin{center}
        \textbf{If no output:}
    \end{center}
    \checkitem{Alt switch}{off}
    \checkitem{Reduce electrical load and land as soon as practical}{}
\end{checklist_emerg}

\textbf{Note:} Checklist is a WIP. Missing emergency procedures (like engine failure) as per 14 CFR § 91.503.

\onecolumn

\phantomsection
\addcontentsline{toc}{section}{Tables and Figures}

\input{../shared/climb_descent_table.tex}
\input{../shared/additional_runway_length_table.tex}

\newcommand{\bookemptyweight}{1634}
\newcommand{\maxgrossweight}{2550}
% Approximately two 160 lb adults, 10 lb of baggage, and 40 gal of fuel remaining.
\newcommand{\grossweight}{2150}
% The calibrated airspeeds for Vx, Vbg, and Vmd–as well as stall speed Vs and
% minimum descent rate (h_md) are porportional to gross aircraft weight.
\newcommand{\scaledvspeed}[1]{\fpeval{#1 + #1 / (2 * \grossweight) * (\grossweight - \maxgrossweight)}}
% Banking transforms these V speeds as well.
\newcommand{\bankedvspeed}[2]{\fpeval{#1 / sqrt(cosd(#2))}}
% Piper's AFM states that "linear interpolation may be used for intermediate
% gross weights."
\newcommand{\maneuveringspeed}{\fpeval{89 + (\grossweight - \bookemptyweight) * ((113 - 89) / (\maxgrossweight - \bookemptyweight))}}
% Keep in mind that there may be some error here–the indicated airspeeds for the
% V speeds mentioned above vary with air density, too (slightly).
\newcommand{\bestglidespeed}{\scaledvspeed{76}}

\phantomsection
\addcontentsline{toc}{subsection}{Archer flight maneuver entry speeds at \num{\grossweight} lbf}

\begin{table}[H]
    \caption{Archer flight maneuver entry speeds at \num{\grossweight} lbf}

    \begin{center}
        \begin{tabular}{lc}
            \toprule
            \textbf{Maneuver} & \textbf{KIAS}                                                 \\
            \midrule
            % We round maneuvering speed down to the nearest five knots because
            % the ACS says we need "an airspeed not to exceed Va."
            Steep Turns       & \fpeval{floor(\maneuveringspeed / 5) * 5}                     \\
            % The Airplane Flying Handbook says to establish "gliding speed"–not
            % any particular "best glide speed." According to Dr. John Lowry's
            % book, Performance of Light Aircraft, "for most (not all) intents
            % and purposes, banking to angle Φ is tantamount to increasing gross
            % weight from W to W / cos Φ." Essentially, banking transforms Vx,
            % Vbg, and Vmd precisely as it does stall speed. In an analysis I
            % performed for the Dakota, I discovered several interesting
            % relationships:
            %   1) Wings-level best glide speed, calculated for aircraft gross
            %      weight, is only 1-2 knots above the stall speed at 45° of
            %      bank.
            %   2) The max gross weight, wings-level best glide speed (e.g.,
            %      what's in the POH) gives an approximate 10 knot margin
            %      against the banked stall speed if flown roughly 400 lbf below
            %      max gross weight.
            %   3) Flying a steep spiral at the max gross weight, wings-level
            %      best glide speed doesn't make sense, even despite the margin.
            %      In a 50° bank, best glide speed increases too. In the Archer,
            %      at 2,150 lbf and 45° of bank, best glide speed is
            %      approximately 86 KCAS. It makes sense to be ahead of L/D max,
            %      so we round up to the nearest 5 knots:
            Steep Spiral      & \fpeval{(ceil(\bankedvspeed{\bestglidespeed}{50} / 10) * 10)} \\
            Chandelles        & \fpeval{floor(\maneuveringspeed / 5) * 5}                     \\
            Lazy Eights       & \fpeval{floor(\maneuveringspeed / 5) * 5}                     \\
            Eights on Pylons  & \fpeval{floor(\maneuveringspeed / 5) * 5}                     \\
            \bottomrule
            % TODO: Add accelerated stall entry speed.
        \end{tabular}
    \end{center}

    \textbf{Note:}
    \begin{itemize}
        \item Design maneuvering speed ($V_A$) at \num{\grossweight} lbf gross weight is $\approx$ \num{\fpeval{floor(\maneuveringspeed, 1)}} KIAS.
        \item Wings-level best glide speed ($V_{bg}$) at \num{\grossweight} lbf gross weight is $\approx$ \num{\fpeval{ceil(\bestglidespeed, 1)}} KIAS.
    \end{itemize}
\end{table}

\input{../shared/speed_vs_pivotal_altitude_table.tex}


\end{document}
