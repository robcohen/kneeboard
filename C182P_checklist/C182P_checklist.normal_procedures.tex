% This is the box that surrounds the checklist items.
\definecolor{ceruleanblue}{rgb}{0.16, 0.32, 0.75}
\newtcolorbox{checklist}[1]{
    colback=white,
    colframe=ceruleanblue,
    fonttitle=\centering\bfseries,
    adjusted title={#1},
    sharpish corners,
    phantom=\phantomsection,
    add to list={toc}{subsection}
}
\definecolor{ao(english)}{rgb}{0.0, 0.5, 0.0}
\newtcolorbox{checklistg}[1]{
    colback=white,
    colframe=ao(english),
    fonttitle=\centering\bfseries,
    adjusted title={#1},
    sharpish corners,
    phantom=\phantomsection,
    add to list={toc}{subsection}
}

\phantomsection
\addcontentsline{toc}{section}{Normal Procedures}

\twocolumn

\begin{checklist}{Preflight}
    \begin{center}
        \emph{Dirty Work}
    \end{center}
    \checkitem{Fuel Sump}{Both Clear}
    \checkitem{Dip Fuel}{Record}
    \checkitem{Engine Oil}{10-12 Qts}
    \begin{center}
        \emph{In-Cabin}
    \end{center}
    \checkitem{Documents}{ARROW}
    \checkitem{Tach/Hobbs}{Recorded}
    \checkitem{Control Lock}{Removed}
    \checkitem{Emergency Equipment}{Check}
    \checkitem{Magnetos}{Off}
    \checkitem{Alternate Static}{Closed}
    \checkitem{Circuit Breakers}{In}
    \checkitem{Electrical Equipment}{Off}
    \checkitem{Bat. Switch}{On}
    \checkitem{Fuel Quantity}{Set}
    \checkitem{Flaps}{Full}
    \checkitem{Avionics/Fan}{On,Fan,Off}
    \checkitem{Bat. Switch}{Off}
    \begin{center}
    \emph{Exterior Inspection}
    \end{center}
    \checkitem{Walk Around}{Complete}
    \checkitem{Engine Fuel Flush}{No Water}
    \checkitem{Tire Pres. Nose/Main}{49/42 PSI}
\end{checklist}

\begin{checklist}{Before Start}
    \checkitem{Tiedowns/Chocks}{Out}
    \checkitem{Towbar}{Stowed}
    \checkitem{Baggage Door}{Secured}
    \checkitem{Passenger Briefing}{Standard}
    \checkitem{Seats/Seat Belts}{Set, Secure}
    \checkitem{Parking Break}{Set}
    \checkitem{Circuit Breakers}{Check}
    \checkitem{Avionics}{Off}
    \checkitem{Fuel Selector}{Both}
    \checkitem{Cowl Flaps}{Open}
\end{checklist}


\begin{checklist}{Start}
    \checkitem{Carburetor Heat}{Cold}
    \checkitem{Throttle}{Open 1/2", Set}
    \checkitem{Propellor}{High RPM, Set}
    \checkitem{Mixture}{Rich, Set}
    \checkitem{Battery Master}{On}
    \checkitem{Beacon}{On}
    \checkitem{Ext. Lights}{On as Required}
    \checkitem{Prime}{As Required}
    \checkitem{Prop. Area}{Clear Prop}
    \checkitem{Ignition}{Start}
    \checkitem{Oil Pressure}{Green 30s/60s}
    \checkitem{Ammeter}{Check, On, Charge}
    \checkitem{Avionics}{On, Set}
    \checkitem{Flaps}{Retract}
    \checkitem{Transponder}{ALT}
    \checkitem{Parking Break}{Off}
    \checkitem{Breaks}{Test}
\end{checklist}

\begin{checklist}{Ready to Taxi}
    \checkitem{Garmin Database}{Updated}
    \checkitem{ATIS}{Copied}
    \checkitem{Transponder}{Set}
    \checkitem{COM \& NAV}{Set}
    \checkitem{Initial Alt.}{Set}
    \checkitem{Initial Heading}{Set}
    \checkitem{Exterior Lights}{Set}
    \checkitem{Clearance}{Recieved}
\end{checklist}

% Mike Busch and John Deakin have made some interesting points about the
% engine run up process. We're running so rich, and at such low power on
% the ground, it's hard to identity any real problems with the ignition
% system outside of a dead spark plug, a dead mag, or severe spark plug
% fouling.
% TBD: Finish this.
% See: https://www.avweb.com/flight-safety/pelicans-perch-19putting-it-all-together/
\begin{checklist}{Engine Run-Up}
    \checkitem{Seats/Belts}{Secure}
    \checkitem{Cabin Doors}{Closed}
    \checkitem{Flight Controls}{Free \& Correct}
    \checkitem{Autopilot}{Check, Off}
    \checkitem{Flight Instruments\*\*}{Set}
    \checkitem{Fuel Quantity}{Check}
    \checkitem{Cowl Flaps}{Open}
    \checkitem{Fuel Selector}{Both}
    \tcblower
    \checkitem{Mixture}{Full Rich}
    % John Deakin says:
    % > The usual 1,700 RPM for running up most TCM engines (or 2,000
    % > RPM for most Lycomings) is NOT critical. I’ve seen pilots diddle
    % > and dawdle trying to get exactly 1,700 but all this does is heat
    % > the engine up for no good purpose. Plus or minus a couple
    % > hundred RPM won’t hurt a thing, so push it up to “about 1,700”
    % > or “about 2,000” and get on with it.
    % > [...]
    % > It is also becoming very clear that the mag check at low power
    % > (anything less than cruise power) is not very useful for
    % > catching problems; it’s nothing more than a quick check to catch
    % > major problems like severe plug fouling, a “hot mag,” or a dead
    % > plug, or cylinder. This was well-known in the big old radials,
    % > where mag checks are almost always performed at about 30″ MP,
    % > and up around 2,300 RPM (varies with model).
    \checkitem{Propellor}{High RPM}
    \checkitem{Throttle}{1700-2000 RPM}
    \checkitem{Oil Pressure/Temp}{Green}
    \checkitem{Cyl. Head Temp}{Green}
    \checkitem{Ammeter}{Check}
    \checkitem{Annunciators}{Check}
    \checkitem{Vacuum}{4.6-5.4 Hg.}
    \checkitem{Magnetos}{Check R \& L}
    \centering{
        (max drop 150; max $\Delta$ 50)
        \\
    }
    \checkitem{Propellor}{Cycle 3X}
    \checkitem{Carb Heat}{Hot}
    \checkitem{Throttle}{Idle}
    \checkitem{Throttle}{700 RPM}
    \checkitem{Carb Heat}{Cold}
    \checkitem{Mixture}{Lean for Taxi}
    \checkitem{Circuit Breakers}{In}
    \checkitem{Alternate Static}{Check}
\end{checklist}

\begin{checklist}{Before Takeoff}
    \checkitem{Doors \& Windows}{Secured}
    \checkitem{Carb. Heat}{Off}
    \checkitem{Flaps}{0-20\degree}
    \checkitem{Trim}{Set}
    \checkitem{Cowl Flaps}{Full Open}
    \checkitem{Lights}{As Req.}

    \begin{center}
        \emph{\hypertarget{departure-briefing}{Departure Briefing}}
    \end{center}

    % TODO: Include some prompt to think about any possibility of
    % tailwind, extreme crosswind, or extreme temperature inversion.
    \checkitem{\hyperlink{runway-length-table}{Takeoff Distance}}{Briefed}
    \checkitem{Terrain \& Obstacles}{Briefed}
    \checkitem{Takeoff Minimums}{Briefed}
    \checkitem{Departure Procedure}{Briefed}

    \begin{center}
        \emph{Abnormal Operations}
    \end{center}

    % TODO: What will we do in case of a fire? Who will fly in an
    % emergency?
    \checkitem{Rejected Takeoff}{Briefed}
    \checkitem{Engine Power Loss}{Briefed}
    \centering{
        (below \& above $\approx$ 600' AGL)
        \\
    }
\end{checklist}

\begin{checklist}{Takeoff}
    \checkitem{Confirm Runway}{\# Confirmed}
    \checkitem{Target Airspeed}{53-78 KIAS}
    \checkitem{Mixture}{Rich/Target EGT}
    \checkitem{Carb Heat}{Cold}
    \checkitem{Throttle}{Full}
    \checkitem{Rotate}{70 KIAS}
    \checkitem{Flaps}{Retract at 70 KIAS}

    \begin{tcolorbox}[boxsep=0mm,left=0mm,right=0mm,colframe=black,colback=black,sharpish corners] 
        \color{white}
        \centering {
            \textbf{IF I LOSE THE ENGINE,\\I WILL PUSH IMMEDIATELY!}
        }
    \end{tcolorbox}
\end{checklist}

\begin{checklist}{Enroute Climb}
    \checkitem{Target Airspeed}{87-96 KIAS}
    \checkitem{Power}{23"/2450 RPM}
    \checkitem{Prop}{As Req.}
    \checkitem{Mixture}{Rich}
    \checkitem{Cowl Flaps}{As Req.}
\end{checklist}

\begin{checklist}{Cruise}
    \checkitem{Target Airspeed}{87-96 KIAS}
    \checkitem{Power}{15"-23"/2200-2450 RPM}
    \checkitem{Prop}{As Req.}
    \checkitem{Mixture}{Leaned}
    \checkitem{Trims}{As Req.}
    \checkitem{Cowl Flaps}{As Req.}
\end{checklist}

\begin{checklist}{Descent}
    \checkitem{Fuel Selector}{Both}
    \checkitem{Cowl Flaps}{As Req.}
    \checkitem{Rudder Trim}{Reset}
    \checkitem{Mixture}{Rich}
    \checkitem{Carb Heat}{As Req.}
    \checkitem{Power}{As Req.}
    \checkitem{ATIS}{Copied} 
    \checkitem{Arrival \& Approach}{Briefed}
    \checkitem{Terrain \& Taxi}{Briefed}
    \checkitem{Specials}{Briefed}
\end{checklist}

\begin{checklistg}{Before Landing}
    \checkitem{Seat \& Belts}{Secure}
    \checkitem{Fuel Selector}{Both}
    \checkitem{Mixture}{Rich}
    \checkitem{Propellor}{High RPM}
    \checkitem{JPI}{Check}
    \checkitem{Rudder Trim}{Neutralize}
    \checkitem{Ext. Lights}{As Req.}
    \checkitem{Pitot Heat}{As Req.}
\end{checklistg}

\begin{checklistg}{Normal Landing}
    \checkitem{Airspeed Flaps Up}{70-78 KIAS}
    \checkitem{Wing Flaps}{0 to 40\degree}
    \checkitem{Airspeed Flaps Down}{61-70 KIAS}
\end{checklistg}

\begin{checklistg}{After Landing}
    \checkitem{Flaps}{Full Retract}
    \checkitem{Cowl Flaps}{Open}
    \checkitem{Carb Heat}{Cold}
    \checkitem{Mixture}{Lean for Taxi}
    \checkitem{Lights}{As Required}
\end{checklistg}

\begin{checklistg}{Securing Aircraft}
    \checkitem{Hobbs \& Tach}{Record}
    \checkitem{Lights}{Off}
    \checkitem{Avionics}{Off}
    \checkitem{Throttle}{700 RPM}
    \checkitem{Mixture}{Idle Cutoff}
    \checkitem{Magnetos}{Off \& Pull Key}
    \checkitem{Master Switch}{Off}
    \checkitem{Position Plane}{Chocks}
    \checkitem{Cowl Flaps}{Closed}
    \checkitem{Parking Break}{Set}
\end{checklistg}

\begin{checklist}{V-Speeds}
    \checkitem{$V_{BG}$ flaps Up/Down}{70/65}
    \checkitem{$V_R$ (flaps 0\degree{}/25\degree{})}{60/50 KIAS}
    \checkitem{$V_{X}$ sea/10K}{59/63 KIAS}
    \checkitem{$V_{Y}$ sea/10K}{80/63 KIAS}
    \checkitem{$V_{A}$}{89-110 KIAS}
    \checkitem{$V_{S_{0}}$/$V_{S_{1}}$}{48/53 KIAS}
\end{checklist}

\onecolumn
