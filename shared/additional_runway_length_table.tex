\phantomsection
\addcontentsline{toc}{subsection}{Additional runway length required to clear low, close-in obstacles}

\begin{table}[H]
    \hypertarget{runway-length-table}{\caption{Additional runway length required to clear low, close-in obstacle}}

    \begin{center}
        \begin{tabular}{lccc}
            \toprule
                                   & \multicolumn{3}{c}{\textbf{Climb Angle}}
            \\\cmidrule(lr){2-4}
                                   & 745'/NM                                  & 530'/NM                                 & 318'/NM                                 \\
            \midrule
            % We subtract 50' from the obstacle height because we assume
            % takeoff performance data clears an initial 50' obstacle.
            \textbf{200' obstacle} & \num{\fpeval{ceil((6076*150)/745, 0)}}'  & \num{\fpeval{ceil((6076*150)/530, 0)}}' & \num{\fpeval{ceil((6076*150)/318, 0)}}' \\
            \textbf{150' obstacle} & \num{\fpeval{ceil((6076*100)/745, 0)}}'  & \num{\fpeval{ceil((6076*100)/530, 0)}}' & \num{\fpeval{ceil((6076*100)/318, 0)}}' \\
            \textbf{100' obstacle} & \num{\fpeval{ceil((6076*50)/745, 0)}}'   & \num{\fpeval{ceil((6076*50)/530, 0)}}'  & \num{\fpeval{ceil((6076*50)/318, 0)}}'  \\
            \bottomrule
        \end{tabular}
    \end{center}

    \textbf{Note:}
    \begin{itemize}
        \item Assumes takeoff performance data is based on clearing a 50' obstacle.
        \item Subtract obstacle's distance from runway end from required runway length.
        \item \hyperlink{departure-briefing}{Return back to the departure briefing.}
    \end{itemize}
\end{table}
