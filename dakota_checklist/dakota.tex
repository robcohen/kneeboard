\documentclass{article}
\usepackage{geometry}
    \geometry{
        a5paper,
        portrait,
        % The Jeppesen plate appears to be closer to 0.25in. I think
        % that 0.5in is looking best for checklists. Compromising to
        % accomodate longer line lengths.
        margin=0.25in,
        rmargin=0.375in,
        % headsep is the separation between header and text. footskip is
        % the separation between baseline of last line of text and
        % baseline of footer. The default is a bit larger. Setting these
        % to the ~line height pleases me, aesthetically.
        headsep=\baselineskip,
        footskip=\baselineskip,
        includehead,
        includefoot
    }
% NASA says:
% > The horizontal spacing between characters should be 25% of the
% > overall size and not less than one stroke width.
%
% The microtype documentation says:
% > Letterspaced fonts for which settings don’t exist will be spaced out
% > by the default of 0.1 em [...]
% AND
% > The amount is specified in thousandths of 1 em [...]
%
% So, we're scaling the default spacing by 25%, and then converting to
% housandths of an em (0.1 * 1000 * .25).
\usepackage[letterspace=25]{microtype}
% Used for the checklist frames.
\usepackage[many]{tcolorbox}
% For the preflight checklist square.
\usepackage{amssymb}
% For finer control over multi-column layouts.
\usepackage{multicol}
% For finer control over headers and footers.
\usepackage{fancyhdr}
% Used for degree symbol.
\usepackage{gensymb}
% Used for PDF ToC links.
\usepackage{hyperref}
% For printing the creation date of the document.
\usepackage{datetime2}
% Used for drawing patterns (e.g., the striped emergency procedure background).
\usepackage{tikz}
\usetikzlibrary{patterns,patterns.meta}
% Used for performance charts.
\usepackage{booktabs}
% Used for performing math inline.
\usepackage{xfp}
% Used for formatting numbers.
\usepackage{siunitx}
\sisetup{
    math-rm=\symup,
    detect-all,
    group-minimum-digits=4,
    group-separator={,}
}
% Improves positioning of tables and figures.
\usepackage{float}

% This is a macro that formats the checklist items and adds a new line.
\def\checkitem#1#2{
    #1\dotfill#2

}

% Set the default font family to sans-serif.
\renewcommand{\familydefault}{\sfdefault}

% NASA says:
% > The vertical spacing between lines should not be smaller than 25-33%
% > of the overall size of the font.
\renewcommand{\baselinestretch}{1.25}

% Configure the header and footer.
\pagestyle{fancy}
\fancyhf{}
\fancyhead[L]{PA-28-236, Piper Dakota}
\fancyfoot[L]{v.\today}
\fancyfoot[R]{\thepage}

% We don't need numbered sections.
\setcounter{secnumdepth}{0}

\begin{document}

% Apply microtype tracking adjustments.
\lsstyle

% [...] you can say \raggedcolumns if you don’t want the bottom lines to
% be aligned. The default is \flushcolumns, so TEX will normally try to
% make both the top and bottom baselines of all columns align.
\raggedcolumns

\include{dakota.preflight_checklist.tex}
% This is the box that surrounds the checklist items.
\newtcolorbox{checklist}[1]{
    colback=white,
    colframe=black,
    fonttitle=\centering\bfseries,
    adjusted title={#1},
    sharpish corners,
    phantom=\phantomsection,
    add to list={toc}{subsection}
}

\phantomsection
\addcontentsline{toc}{section}{Normal Procedures}

\twocolumn

\begin{checklist}{Before Start}
    \checkitem{Walk around}{completed}
    \checkitem{Seat belts}{fastened}
    \checkitem{PIC}{established}
    \checkitem{Passengers}{briefed}
\end{checklist}

\begin{checklist}{Engine Start}
    \checkitem{Parking brake}{set}
    \checkitem{Fuel}{desired tank}
    \checkitem{Carburetor heat}{off}
    \checkitem{Master switch}{on}
    \checkitem{Fuel pump}{on}
    \checkitem{Mixture}{full rich}
    \checkitem{Throttle}{$\frac{1}{4}''$}
    \checkitem{Prime}{5-6 shots, if cold}
    \checkitem{Clear/starter}{engage}
    \checkitem{Throttle}{idle}
    \checkitem{Mixture}{lean for taxi}
    \checkitem{Oil pressure}{check}
    \checkitem{Load meter}{check}
    \checkitem{Fuel pump}{off}
    \checkitem{Fuel pressure}{check}
\end{checklist}

\begin{checklist}{After Start}
    \checkitem{Avionics \& G5 master}{on}
    \checkitem{Circuit breakers}{check}
    \checkitem{Garmin database}{check}
    \checkitem{Garmin self-test}{check}
    \tcblower
    \checkitem{ATIS \& clearance}{recieved}
\end{checklist}

\begin{checklist}{Before Taxi}
    \checkitem{Transponder}{set}
    \checkitem{COM \& NAV}{set}
    \checkitem{Initial altitude}{set}
    \checkitem{Initial heading}{set}
\end{checklist}

\begin{checklist}{Taxi}
    \checkitem{Exterior lights}{set}
    \checkitem{Brakes}{check}
    \checkitem{Heading indicator}{±5°}
    \checkitem{Attitude indicator}{check}
    \checkitem{Turn coordinator}{check}
\end{checklist}

% Mike Busch and John Deakin have made some interesting points about the
% engine run up process. We're running so rich, and at such low power on
% the ground, it's hard to identity any real problems with the ignition
% system outside of a dead spark plug, a dead mag, or severe spark plug
% fouling.
% TBD: Finish this.
% See: https://www.avweb.com/flight-safety/pelicans-perch-19putting-it-all-together/
\begin{checklist}{Engine Run-Up}
    \checkitem{Mixture}{full rich}
    \checkitem{Prop}{full forward}
    % John Deakin says:
    % > The usual 1,700 RPM for running up most TCM engines (or 2,000
    % > RPM for most Lycomings) is NOT critical. I’ve seen pilots diddle
    % > and dawdle trying to get exactly 1,700 but all this does is heat
    % > the engine up for no good purpose. Plus or minus a couple
    % > hundred RPM won’t hurt a thing, so push it up to “about 1,700”
    % > or “about 2,000” and get on with it.
    % > [...]
    % > It is also becoming very clear that the mag check at low power
    % > (anything less than cruise power) is not very useful for
    % > catching problems; it’s nothing more than a quick check to catch
    % > major problems like severe plug fouling, a “hot mag,” or a dead
    % > plug, or cylinder. This was well-known in the big old radials,
    % > where mag checks are almost always performed at about 30″ MP,
    % > and up around 2,300 RPM (varies with model).
    \checkitem{Throttle}{1800 RPM}
    % \checkitem{JPI}{normalize}
    \checkitem{Mags}{check L \& R}
    \centering{
        (max drop 175; max $\Delta$ 50)
        \\
    }
    \checkitem{Carburetor heat}{check}
    % When running up the Lance, Jim Cherry was taken aback when I
    % cycled the propellor at 1800 RPM. His reasoning was that it's
    % overkill, it doesn't need to be cycled that high. Unnessecary
    % strain on things, why do it? But, I forget the ballpark that he
    % reccomended. It may have been 1000-1500 RPM.
    \checkitem{Prop}{cycle}
    \centering{
        ($\downarrow$ rpm $\uparrow$ mp $\downarrow$ oil press.)
        \\
    }
    \checkitem{Vacuum}{4.9-5.1$''$Hg}
    \checkitem{Load meter}{check}
    \checkitem{Fuel pressure}{check}
    \checkitem{Oil pressure \& oil temp.}{check}
    \checkitem{Alternate static}{check}
    \checkitem{Annunciator panel}{check}
    \checkitem{Throttle}{idle}
    \checkitem{Mixture}{lean for taxi}
\end{checklist}

\begin{checklist}{Before Takeoff}
    \checkitem{Flight controls}{check}
    \checkitem{Flight instruments}{check}
    \checkitem{Carburetor heat}{off}
    \checkitem{Flaps}{set}
    \checkitem{Trim}{set}
    \begin{center}
        \emph{Departure Briefing}
    \end{center}
    \checkitem{Takeoff distance}{briefed}
    % TODO: Include some prompt to think about any possibility of
    % tailwind, extreme crosswind, or extreme temperature inversion.
    \checkitem{Low, close-in obstacles}{briefed}
    \checkitem{Departure procedure}{briefed}
    \checkitem{Abnormal operations}{briefed}
\end{checklist}

\begin{checklist}{Takeoff}
    \checkitem{Time off}{noted}
    \checkitem{Doors \& windows}{secured}
    \checkitem{Exterior lights}{set}
    \checkitem{Fuel pump}{on}
    \checkitem{Mixture}{set}
    \checkitem{Prop}{full forward}
    \checkitem{Throttle}{full power}

    \begin{tcolorbox}[boxsep=0mm,left=0mm,right=0mm,colframe=black,colback=black,sharpish corners] 
        \color{white}
        \centering {
            \textbf{I WILL LOSE THE ENGINE,\\I WILL PUSH IMMEDIATELY!}
        }
    \end{tcolorbox}
\end{checklist}

\pagebreak

\begin{checklist}{Before Approach}
    \checkitem{NOTAMS}{briefed}
    \checkitem{ATIS, Arrival, \& Approach}{briefed}
    \checkitem{Terrain \& Taxi}{briefed}
    \checkitem{Specials}{briefed}
\end{checklist}

\begin{checklist}{Approach}
    \checkitem{Altimeter}{verify}
    \checkitem{DA or MDA}{verify MSL}
    \checkitem{Throttle}{13$''$}
    \checkitem{Prop}{2300 RPM}
    \checkitem{Mixture}{set}
    \checkitem{Airspeed}{100 KIAS}
\end{checklist}

\begin{checklist}{After Landing}
    \checkitem{Flaps}{retract}
    \checkitem{Mixture}{lean for taxi}
    \checkitem{Fuel pump}{off}
\end{checklist}

\begin{checklist}{V-Speeds}
    \checkitem{$V_{BG}$}{85 KIAS}
    \checkitem{$V_R$ (flaps 0\degree{})}{77 KIAS}
    \checkitem{$V_R$ (flaps 25\degree{})}{52-63 KIAS}
    \checkitem{$V_{X}$}{73 KIAS}
    \checkitem{$V_{Y}$}{85 KIAS}
    \checkitem{$V_{CC}$}{100 KIAS}
    \checkitem{$V_{Ref}$}{72 KIAS}
    \checkitem{$V_{A}$}{96-124 KIAS}
    \checkitem{$V_{S_{0}}$/$V_{S_{1}}$}{56/65 KIAS}
\end{checklist}
\end{checklist}

\onecolumn

\newtcolorbox{checklist_emerg}[1]{
    coltitle=red!25!black,
    colback=white,
    colframe=red!75,
    fonttitle=\centering\bfseries,
    adjusted title={#1},
    sharpish corners
}

\section{Emergency Procedures}

\pagebreak

\begin{checklist_emerg}{Electrical Fire (Smoke in Cabin)}
    \begin{enumerate}
        \item \checkitem{Master switch}{off}
        \item \checkitem{Avionics master}{off}
        \item \checkitem{Electrical switches}{off}
        \item{
            \textbf{If no smoke:}
            \begin{enumerate}
                \item \checkitem{Circuit breakers}{note tripped}
                \item \checkitem{Circuit breakers}{off}
                \item \checkitem{Master switch}{on}
                \item{
                    \textbf{If no smoke:}
                    \begin{enumerate}
                        \item \checkitem{Avionics master}{on}

                    \end{enumerate}
                }
            \end{enumerate}
        }
    \end{enumerate}
\end{checklist_emerg}

\begin{checklist_emerg}{Alternator Failure}
    \begin{enumerate}
        \item Verify failure.
        \item Reduce electrical load as much as possible.
        \item \checkitem{Alternator circuit breakers}{check}
        \item \checkitem{Alt switch}
                {off (for 1 second), then on}
        \item \textbf{If no output:}
        \begin{enumerate}
            \item \checkitem{Alt switch}{off}
        \end{enumerate}
        Reduce electrical load and land as soon as practical.
    \end{enumerate}
\end{checklist_emerg}

% (1) Emergency operation of fuel, hydraulic, electrical, and mechanical systems.

% (2) Emergency operation of instruments and controls.

% (3) Engine inoperative procedures.

% (4) Any other procedures necessary for safety. 

\phantomsection
\addcontentsline{toc}{section}{Tables and Figures}

\phantomsection
\addcontentsline{toc}{subsection}{Rate of climb/descent (ft. per min)}

\begin{table}[ht]
    \caption{Rate of climb/descent (ft. per min)}

    \begin{center}
        \begin{tabular}{cccc}
            \toprule
                             &       & \multicolumn{2}{c}{Ground speed (knots)}
            \\\cmidrule(lr){3-4}
            Angle            & ft/NM & 60                                       & 90   \\
            \midrule
            3.0\textdegree{} & 318   & 318                                      & 478  \\
            5.0\textdegree{} & 530   & 530                                      & 795  \\
            7.0\textdegree{} & 745   & 745                                      & 1120 \\
        \end{tabular}
    \end{center}
\end{table}

\phantomsection
\addcontentsline{toc}{subsection}{Additional runway length required to clear low, close-in obstacles}

\begin{table}[ht]
    \hypertarget{runway-length-table}{\caption{Additional runway length required to clear low, close-in obstacle}}

    \begin{center}
        \begin{tabular}{lccc}
            \toprule
                          & \multicolumn{1}{c}{745'/NM (7.0\textdegree{})} & \multicolumn{1}{c}{530'/NM (5.0\textdegree{})} & \multicolumn{1}{c}{318'/NM (3.0\textdegree{})}
            \\\cmidrule(lr){2-2}\cmidrule(lr){3-3}\cmidrule(lr){4-4}
                          & Increase by                                    & Increase by                                    & Increase by                                    \\\midrule
            % We subtract 50' from the obstacle height because we assume
            % takeoff performance data clears an initial 50' obstacle.
            200' obstacle & \num{\fpeval{ceil((6076*150)/745, 0)}}'        & \num{\fpeval{ceil((6076*150)/530, 0)}}'        & \num{\fpeval{ceil((6076*150)/318, 0)}}'        \\
            150' obstacle & \num{\fpeval{ceil((6076*100)/745, 0)}}'        & \num{\fpeval{ceil((6076*100)/530, 0)}}'        & \num{\fpeval{ceil((6076*100)/318, 0)}}'        \\
            100' obstacle & \num{\fpeval{ceil((6076*50)/745, 0)}}'         & \num{\fpeval{ceil((6076*50)/530, 0)}}'         & \num{\fpeval{ceil((6076*50)/318, 0)}}'         \\
        \end{tabular}
    \end{center}

    \textbf{Note:}
    \begin{itemize}
        \item Assumes takeoff performance data is based on clearing a 50' obstacle.
        \item Subtract obstacle's distance from runway end from required runway length.
        \item \hyperlink{departure-briefing}{Return back to the departure briefing.}
    \end{itemize}
\end{table}

\phantomsection
\addcontentsline{toc}{subsection}{Dakota flight maneuver entry speeds}

\begin{table}[ht]
    \caption{Dakota flight maneuver entry speeds}

    \begin{center}
        \begin{tabular}{lc}
            \toprule
            Maneuver         & KIAS \\
            \midrule
            Steep Turns      & 110  \\
            Steep Spiral     & 85   \\
            Chandelles       & 110  \\
            Lazy Eights      & 110  \\
            Eights on Pylons & 110  \\
            % TODO: Add accelerated stall entry speed.
        \end{tabular}
    \end{center}

    % TODO: Maneuvering speed table.
    \textbf{Note:} Maneuvering speed at 2,500 lbf aircraft gross weight is 111.8 KIAS.
\end{table}


\end{document}
